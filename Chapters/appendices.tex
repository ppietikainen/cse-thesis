% Appendix Chapter

Appendix 1 \hskip 2.5em Example Abstract 
\vspace{7em}

\noindent Note 1: Page numbering is done by consecutive numbering. Additional appendices, that are attached to the thesis (such as copies, drawings, etc.) are left without page numbers and placed at the end of the thesis.

\newpage
\markboth{left}{\normalfont{Appendix 1. Example Abstract}}

\noindent \textbf{Jurmu M.\ (2007) Resource Management in Smart Spaces Using Context-Based Leases.} University of Oulu, Department of Electrical and Information Engineering. Master’s Thesis, 77 p.

\begin{center}
\textbf{\fontsize{16}{19pt}\selectfont ABSTRACT}\\
\end{center}

{\bfseries \noindent The convergence of wireless access networks in conjunction with the increased computing power of the handheld terminals is preparing the emergence of the ubiquitous computing paradigm. In the center of this paradigm are smart spaces, which are local environments saturated with various embedded computational resources. These spaces co-operate with mobile client devices in enabling advanced, service-oriented computation scenarios. This co-operation is typically enabled through the utilization of distributed and modular middleware frameworks. An emerging additional requirement however is the possibility to harness resources from the proximity environment to the mobile device in an on-demand fashion. This is a challenge especially to the resource management infrastructure of the smart spaces.

This thesis explores the concept of a smart space and presents a review of the current research and technologies. Subsequently, a lease-based design for resource management in smart spaces is presented. Leases in this work are negotiated agreements between the mobile clients and the resource management infrastructure, regarding the harnessed resources. Leasing is seen as a suitable solution for the management due to the transient nature of the resource usage. The inclusion of additional contextual features to the leases further facilitates the management.

Requirements for the design are derived from the review and from an example usage scenario. The requirements include dynamic mapping and contracting of resources from the proximity environment, monitoring of the contract validity, access control towards the resources and dynamic maintenance of the smart space infrastructure. Presented design is analytically compared against existing solutions, and several points for future development are listed. According to the comparison, none of the existing solutions utilize contracts with contextual validity in resource management. Two publications of this work have been accepted into an international conference and a workshop on the focus area of pervasive computing.

\vspace{1\baselineskip}
\noindent\textbf{Keywords: Ubiquitous computing, mobile computing, task-based computing, context-awareness, QoS.}
}
